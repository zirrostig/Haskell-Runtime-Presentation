%%%presentation%%%\documentclass[aspectratio=169]{beamer}
%%%handout%%%\documentclass[handout]{beamer}
%%%notes%%%\documentclass[handout]{beamer}

\usetheme{CambridgeUS}
\usecolortheme{lily}
\usepackage[latin1,utf8]{inputenc}
\usepackage{verbatim}
\usepackage{hyperref}
\usepackage{pgfpages}
\usepackage{amssymb,amsmath}
\usepackage{minted} %using patched version included in dir, provides autogobble option
\usepackage{hscode}
\usemintedstyle{tango}

%Haskell code environment found in hsenv.sty
\newminted[hsframe]{haskell}{frame=single,linenos,autogobble,xleftmargin=1cm}
\newmint[hsline]{haskell}{}
\makeatletter
\newenvironment<>{hscode}
    {\VerbatimEnvironment
     \minted@resetoptions
     \setkeys{minted@opt}{frame=single,linenos,mathescape,autogobble,xleftmargin=1cm}
     \begin{onlyenv}#1
     \begin{VerbatimOut}{\jobname.pyg}
    }
    {\end{VerbatimOut}
     \minted@pygmentize{haskell}
     \end{onlyenv}
     \DeleteFile{\jobname.pyg}
    }
\makeatother

%To disable double wide presentation for notes
%%%withnotes%%%\setbeameroption{show notes on second screen}

%To Generate Notes
%%%notes%%%\setbeameroption{show only notes}
%%%notes%%%\pgfpagesuselayout{2 on 1}[letterpaper, border shrink=1cm]

%To Generate handouts
%%%handout%%%\pgfpagesuselayout{2 on 1}[letterpaper,border shrink=1cm]
\title{Haskell at Runtime}
\author{Zachary Stigall}
\date{\today}
\institute[\the\year]{Boulder Haskell Programmers}

\begin{document}
% \setbeamertemplate{caption}{\insertcaption}

\maketitle

% \begin{frame}[allowframebreaks,t]{What Are We Covering}
%     \tableofcontents
% \end{frame}

% \section{Definitions}
% \begin{frame}[t]{Thunk}
%     \begin{definition}
%         A \alert{Thunk} is something that is yet to be evaluated.
%     \end{definition}
% \end{frame}

% \begin{frame}[t]{Weak Head Normal Form}
%     \begin{definition}
%         We know what it looks like, but haven't evaluted it fully.
%     \end{definition}
%     \begin{figure}
%         \includegraphics[width=.4\paperwidth]{../img/whnf.png}
%         \caption{Source: \url{http://en.wikibooks.org/wiki/Haskell/Laziness}}
%     \end{figure}
% \end{frame}

\section{GHC}
\begin{frame}[t]{What Does GHC Do?}
    \begin{itemize}
        \item<2-> Desugars Haskell code into Core.
        \item<5-> Applies Optimizations to Core
        \item<6-> Converts Core into STG
        \item<7-> Compiles STG to C-{}- (kinda)
        \item<8-> C-{}- is then compiled to LLVM / C / Machine Code
        \item<9-> Packs on the Haskell Runtime System (RTS)
    \end{itemize}

    \begin{definition}<3->
        \alert{Core} is GHC's intermediate language based on System FC, with a couple additions.
    \end{definition}

    \begin{definition}<4->
        \alert{System F} is a Polymorphic Lambda Calculus language, \alert{FC} adds GADTS to it.
    \end{definition}
\end{frame}

\section{RTS}
\subsection{Overview}
\begin{frame}[t]{RTS}
    The RTS provides:
    \begin{itemize}[<+->]
        \item Garbage Collection
        \item Scheduler
        \item Execution of non-compiled code
        \item Dynamic Linker
        \item Profiler
        \item Software Transactional Memory
    \end{itemize}
\end{frame}

\subsection{GC}
\begin{frame}[t]{GC}
    \begin{itemize}[<+->]
        \item Very Efficient - Generational (Out of necessity)
        \item Data does not point to younger generations
        \item This lets the GC just remove all non-pointed to data at GC
    \end{itemize}

    \uncover<+->{Links:\\
        \href{http://www.haskell.org/haskellwiki/GHC/Memory_Management}{Broad Overview}\\
        \href{https://ghc.haskell.org/trac/ghc/wiki/Commentary/Rts/Storage}{Way too in depth (ghc.haskell.org)}
    }
\end{frame}

\begin{frame}[t]{GC - Flags}
    \uncover<+->{These require program is compiled with `-rtsopts'.\\}
    \uncover<+->{This does not cover all options, just the most common.}
    \begin{itemize}[<+->]
        \item \alert{-A[size]} sets allocated area for GC (default 512K)
        \item \alert{-H[size]} suggested Heap size. (size optional)
        \item \alert{-c[n]} does compacting, use only when you need to drastically reduce RAM usage
        \item \alert{-qg} parallel GC
    \end{itemize}

    \uncover<+->{Recommended \alert{ghc-gc-tune} - Allows profiling of varying -A and -H parameters}
\end{frame}

\subsection{Scheduler}
\begin{frame}[t,fragile]{Scheduler}
    Horribly Complicated
    \begin{itemize}[<+->]
        \item Centered around the Run Queue
        \item The Run Queue is a dispatcher for Capabilities
        \item Capabilities are virtual CPU's that execute Thread State Objects (TSO)
        \item TSO's are essentially closures and are GC'ed
    \end{itemize}

    \uncover<5->{Links:\\
        \href{https://ghc.haskell.org/trac/ghc/wiki/Commentary/Rts/Scheduler}{Very Detailed (ghc.haskell.org)}\\
        \href{http://blog.ezyang.com/2013/01/the-ghc-scheduler/}{Nice blog post by Edward Z. Yang}
    }
\end{frame}

\subsection{Profiling}
\begin{frame}[t]{Basic Profiling}
    \begin{itemize}
        \item<1-> All profiling will require the program compiled with \alert{-rtsopts}
        \item<2-> For some basic (but very detailed) profiling compile with \alert{-prof -fprof-auto -rtsopts}
        \item<3-> Run with Prog [ARGS] +RTS [RTS-OPTS]
        \item<4-> \alert{-s} gives a summary
        \item<5-> \alert{-p} writes Prog.prof with tons of details
    \end{itemize}
    \uncover<6->{Demo\\}

    \begin{itemize}
        \item<7-> You can also tell GHC that you want analysis on a specific part by inserting your own cost center
        \item<8-> \alert{\{-\# SCC "CC-Name" \#-\} (expression)}
    \end{itemize}
    \uncover<9->{More Demo\\}
\end{frame}

\begin{frame}[t]{Heap Profiling}
    \uncover<+->{Same compile opts as last time\\}
    \uncover<+->{\alert{-hy} as an RTS-OPT, produces a Prog.hp, this can be converted to postscript with hp2ps\\}
    \uncover<+->{Demo\\}
\end{frame}

\begin{frame}[t]{Other Profiling}
    \begin{itemize}[<+->]
        \item \alert{-B} Sound Bell at GC
        \item \alert{-xc} Print Stack Trace on exception
        \item \alert{-M} Set Max Heap size
    \end{itemize}
\end{frame}

\section{Closing}
\begin{frame}[t]{That's It}
    Questions? \\
    My GitHub: \url{http://github.com/ZirroStig}
\end{frame}

\end{document}
